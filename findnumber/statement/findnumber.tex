\documentclass[a4paper,11pt]{article}

\usepackage[a4paper, left=2.5cm, right=2.5cm, bottom=2.5cm, top=3.5cm,
headheight=1cm]{geometry}
\usepackage{polyglossia}
\usepackage{amsmath}
\usepackage{fancyhdr}
\usepackage{titlesec}
\usepackage{enumitem}


\setmainfont[Ligatures=TeX]{Noto Serif}
\newfontfamily\cyrillicfont[Script=Cyrillic]{Noto Serif}

\newcommand{\code}[1]{\texttt{#1}}
\newcommand{\noskip}{\vspace{-\parskip}}

\setmainlanguage{bulgarian}
\PolyglossiaSetup{bulgarian}{indentfirst=true}

\setlength{\parindent}{0em}
\setlength{\parskip}{1em}
\titlespacing{\section}{0pt}{0pt}{0pt}
\setlist{nosep}

\renewcommand{\labelitemi}{$\bullet$}
\renewcommand{\labelitemii}{$\circ$}
\renewcommand{\thesection}{}

\pagestyle{fancy}
\fancyhf{}
\rfoot{\thepage}
\chead{\uppercase{Национална школа по информатика \\Стара Загора, 24-30 август
2017 г.}}
\renewcommand{\headrulewidth}{0pt}

\title{Открий числото}
\author{Александър Кръстев}
\date{}

\makeatletter
\begin{document}
{%
    \centering \LARGE 
    \textbf{\@title}
    \par
}
%\vspace{0.5cm}

Ави и Боби играят следната игра:
Ави си е намисля редица от $N$ числа $a_0,a_1,...a_{N-1}$, подредени в строго
нарастващ ред, т.е. $a_i < a_{i+1}$ за всяко $0 \leq i \leq N-2$.
След това тя избира едно число от редицата $a_k$ и го съобщава на Боби.
Боби трябва да намери индекса на числото (т.е $k$) като може да задава
на Ави въпроси от вида „Каква е стойността на $a_i$ ?“ за избрана от него
стоиност $i$.

Вашата задача е да напишете програма, играеща ролята на Боби, която да намира
индекса $k$ на избраното от Ави число $a_k$, задавайки най-много $Q$ въпроса.


\section*{Детайли по имплментацията}
Вашата програма трябва да съдържа в началото си \code{\#include "findnumber.h"}.
\\
Вашата програма трябва да реализира следната функция:

\noskip
\begin{itemize}
    \item \code{int find\_index(int N, int a)}
    \begin{itemize}
        \item \code{N}: дължината на намислената от Ави редица
        \item \code{a}: стойността на избраното от Ави число $a_k$
        \item Функцията трябва да връща стойността на индекса $k$.
        \item Функцията се вика веднъж за всеки тест.
    \end{itemize}
\end{itemize}

Вашата програма може да задава върпоси, използвайки следната функция:
\noskip
\begin{itemize}
    \item \code{int get\_number(int i)}
    \begin{itemize}
        \item \code{i}: индексът на числото, за което Вашата програма пита.
            $i$ трябва да е валиден индекс, т.е $0 \leq i \leq N - 1$.
        \item Функцията връща стойността на $a_i$.
        \item Функцията може да се вика $Q$ пъти за един тест.
    \end{itemize}
\end{itemize}


В случай, че наруши ограниченията за параметъра на \code{get\_number}, ще
получите съобщение „Invalid question asked.“.

В случай, че извикате функцията \code{get\_number} повече от $Q$ пъти, ще
получите съощение „Too many questions asked.“.

\section*{Пример}
Грейдърът извършва следното извикване на функция:
\noskip
\begin{itemize}
    \item \code{find\_index(5, 3)} Ави е намислилиа редица с $N=5$ елемента и
        Боби търси индекса на числото със стойност $3$.
\end{itemize}
Програмата извършва следните извиквания на функции:
\noskip
\begin{itemize}
    \item \code{get\_number(0)} връща 0
    \item \code{get\_number(3)} връща 7
    \item \code{get\_number(1)} връща 1
    \item \code{get\_number(2)} връща 3
    \item \code{get\_number(4)} връща 5
\end{itemize}

Редицата на Ави е $0, 1, 3, 5, 7$ и индексът на търсената стойност $3$ е $2$.
Съответно \code{find\_index} трябва да върне $2$.

\section*{Подзадачи}
Във всички подзадачи $0 \leq a_i \leq 10^9$ за всяко $0 \leq i \leq N - 1$.

Нека означим с $Q$ максималния разрешен брой извиквания на \code{get\_number} за
едно изпълнение на програмата.
\noskip
\begin{enumerate}
    \item (50 точки) $1 \leq N \leq 100\ 000$, $Q = 100\ 000$
    \item (50 точки) $1 \leq N \leq 100\ 000$, $Q = 17$
\end{enumerate}
\section*{Локално тестване}
За да можете да тествате решението си на компютъра си, Ви се предоставят
файловете \code{Lgrader.cpp} и \code{findnumber.h}, които да компилирате
заедно с Вашето решение \code{findnumber.cpp}.

\noskip
\subsection*{Вход}
\noskip
\begin{itemize}
    \item Ред 1: три цели числа $N$, $k$ и $Q$
    \item Ред 2: $N$ числа $a_0, a_1, ..., a_{N-1}$
\end{itemize}
\noskip
\subsection*{Изход}
\noskip
\begin{itemize}
    \item Ред 1:
    \begin{itemize}
        \item \textit{„Output is correct.“},
            ако програмата е преминала успешно теста.
        \item \textit{„Invalid question.“},
            ако пограмата е задала въпрос за стойността на невалиден индекс $i$.
        \item \textit{„Too many questions asked.“},
            ако програмата е извършила повече от $Q$ извиквания на
            \code{get\_number}.
        \item \textit{„Output isn't correct.“},
            ако програмата не е надвишила максималния разрешен брой въпроси, но
            не е намерила правилно търсения индекс $k$.
    \end{itemize}
\end{itemize}

\section*{Изпращане на тестове към системата}
Можете да изпращате собствени тестове към системата. Форматът на входните и
изходните данни е същият като този на предоставения локалнен грейдър.
\end{document}
\makeatother
