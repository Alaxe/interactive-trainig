\documentclass[a4paper,11pt]{article}

\usepackage[a4paper, left=2.5cm, right=2.5cm, bottom=2.5cm, top=3.5cm,
headheight=1cm]{geometry}
\usepackage{polyglossia}
\usepackage{amsmath}
\usepackage{fancyhdr}
\usepackage{titlesec}
\usepackage{enumitem}


\setmainfont[Ligatures=TeX]{Noto Serif}
\newfontfamily\cyrillicfont[Script=Cyrillic]{Noto Serif}

\newcommand{\code}[1]{\texttt{#1}}
\newcommand{\noskip}{\vspace{-\parskip}}

\setmainlanguage{bulgarian}
\PolyglossiaSetup{bulgarian}{indentfirst=true}

\setlength{\parindent}{0em}
\setlength{\parskip}{1em}
\titlespacing{\section}{0pt}{0pt}{0pt}
\setlist{nosep}

\renewcommand{\labelitemi}{$\bullet$}
\renewcommand{\labelitemii}{$\circ$}
\renewcommand{\thesection}{}

\pagestyle{fancy}
\fancyhf{}
\rfoot{\thepage}
\chead{\uppercase{Национална школа по информатика \\Стара Загора, 24-30 август
2017 г.}}
\renewcommand{\headrulewidth}{0pt}

\title{Заявки за суми}
\author{Александър Кръстев}
\date{}

\makeatletter
\begin{document}
{%
    \centering \LARGE 
    \textbf{\@title}
    \par
}
%\vspace{0.5cm}

Ави има редица от $N$ цели числа $a_0,a_1,...,a_{N-1}$.
Тя се интересува от сумите на определени подинтервали на редицата.
Сумата на подинтервал $[i, j]$ ($0 \leq i \leq j \leq N-1$) е равна на 
$a_i + a_{i+1} + ... + a_j$.
Тъй като Ави е много заета това лято, тя не може да извърши пресмятанията сама и
затова Ви моли да ѝ помогнете.

Вашата задача е да напишете програма, която по зададена редица 
$a_0, a_1, ..., a_{N-1}$ отговаря на заявки за суми на нейни подинтервали.


\section*{Детайли по имплментацията}
Вашата програма трябва да съдържа в началото си \code{\#include "sumqueries.h"}.
\\
Вашата програма трябва да реализира следните две функции:

\noskip
\begin{itemize}
    \item \code{void init(std::vector<int> a)}
    \begin{itemize}
        \item \code{a}: редицата на Ави.
        \item Функцията не трябва да връща никаква стойност.
        \item Функцията се вика веднъж за даден тест, в началото на изпълнението
            на програмата.
    \end{itemize}
    \item \code{long long sum(int i, int j)}
    \begin{itemize}
        \item \code{i} и \code{j}: началато и краят на подинтервала, чиято сума
        се търси. Гарантирано е, че $0 \leq i \leq j \leq N-1$.
        \item Функцията трябва да връща сумата $a_i + a_{i+1} + ... + a_j$.
        \item Функцията може да се вика много пъти за един тест, но винаги след
            \code{init}.
    \end{itemize}
\end{itemize}

\section*{Пример}
Първо грейдърът извършва следото извикване на \code{init}
\noskip
\begin{itemize}
    \item \code{init([1, 8, -1, 3, 5])} Редицата на Ави е $1, 8, -1, 3, 5$.
\end{itemize}
След това грейдърът извършва следните извиквания на \code{sum}
\noskip
\begin{itemize}
    \item \code{sum(0, 0)} трябва да върне 1.
    \item \code{sum(0, 2)} трябва да върне 8.
    \item \code{sum(2, 4)} трябва да върне 7.
    \item \code{sum(0, 4)} трябва да върне 16.
    \item \code{sum(1, 3)} трябва да върне 10.
\end{itemize}

\section*{Подзадачи}
Във всички подзадачи $-10^9 \leq a_i \leq 10^9$ за всяко $0 \leq i \leq N - 1$.

Нека означим с $Q$ броя на извикванията на Вашата функция \code{sum} от
грейдъра.
\noskip
\begin{enumerate}
    \item (50 точки) $1 \leq N \leq 1\ 000$, $1 \leq Q \leq 1\ 000$
    \item (50 точки) $1 \leq N \leq 200\ 000$, $1 \leq Q \leq 200\ 000$
\end{enumerate}
\section*{Локално тестване}
За да можете да тествате решението си на компютъра си, Ви се предоставят
файловете \code{Lgrader.cpp} и \code{sumqueries.h}, които да компилирате
заедно с Вашето решение \code{sumqueries.cpp}.

\noskip
\subsection*{Вход}
\noskip
\begin{itemize}
    \item Ред 1: две цели числа $N$ и $Q$
    \item Ред 2: $N$ числа $a_0, a_1, ..., a_{N-1}$
    \item Следващите $Q$ реда: по две цели числа $i$ и $j$, началото и краят на
        подинтервала на съответната заявка
\end{itemize}
\noskip
\subsection*{Изход}
\noskip
\begin{itemize}
    \item $Q$ реда: по едно цяло число, отговора на съответната заявка.
\end{itemize}

\section*{Изпращане на тестове към системата}
Можете да изпращате собствени тестове към системата. Форматът на входните и
изходните данни е същият като този на предоставения локалнен грейдър.
\end{document}
\makeatother
