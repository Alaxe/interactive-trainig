\documentclass[a4paper,11pt]{article}

\usepackage[a4paper, left=2.5cm, right=2.5cm, bottom=2.5cm, top=3.5cm,
headheight=1cm]{geometry}
\usepackage{polyglossia}
\usepackage{amsmath}
\usepackage{fancyhdr}
\usepackage{titlesec}
\usepackage{enumitem}


\setmainfont[Ligatures=TeX]{Noto Serif}
\newfontfamily\cyrillicfont[Script=Cyrillic]{Noto Serif}

\newcommand{\code}[1]{\texttt{#1}}
\newcommand{\noskip}{\vspace{-\parskip}}

\setmainlanguage{bulgarian}
\PolyglossiaSetup{bulgarian}{indentfirst=true}

\setlength{\parindent}{0em}
\setlength{\parskip}{1em}
\titlespacing{\section}{0pt}{0pt}{0pt}
\setlist{nosep}

\renewcommand{\labelitemi}{$\bullet$}
\renewcommand{\labelitemii}{$\circ$}
\renewcommand{\thesection}{}

\pagestyle{fancy}
\fancyhf{}
\rfoot{\thepage}
\chead{\uppercase{Национална школа по информатика \\Стара Загора, 24-30 август
2017 г.}}
\renewcommand{\headrulewidth}{0pt}

\title{Топ секретно}
\author{Александър Кръстев}
\date{}

\makeatletter
\begin{document}
{%
    \centering \LARGE 
    \textbf{\@title}
    \par
}
%\vspace{0.5cm}
На Ави ѝ е възложена топ секретна мисия, при изпълнението на която момичето
трябва да се представя за служител в дадена фирма
(детайлите на операцията са строго поверителни).
Във фирмата работят $N+1$ служители, номерирани $0, 1, ..., N$.
Ави се представя за служител номер $N$, който тя се е погрижила да не е на
работа.
Всичко върви гладко, докато не идва време за обяд…

Служителите във фирмата се нареждат на опашката за храна в точно определен ред.
Нека означим с $a_i$ кой подред се нарежда $i$-тият служител.
Разбира се, редицата $a_0, a_1, ..., a_N$ е пермутация, т.е. $a_i \neq a_j$ 
за всички $0 \leq i < j \leq N$ и $0 \leq a_i \leq N$ за всяко
$0 \leq i \leq N$.
Проблемът е, че Ави не знае този ред и съответно не знае къде в опашката да
застане.
Тя трябва да заеме място $a_N$ (мястото на служител $N$) - ако не го стори,
рискува да бъде разкрита.

За радост Ави не е в пълна безизходица, тъй като тя може да извлича информация
от другите служители.
Момичето може да пита служител $i$ дали е сред първите $k$ души в опашката, т.е.
дали $a_i < k$ ($0 \leq i \leq N - 1$).
Разбира се, тя не може да задава въпроси на служител номер $N$, защото него го
няма.
За да не буди подозрения, Ави може да зададе най-много $Q$ такива въпроса.

Вашата задача е да напишете програма, която помага на Ави да открие стойността
на $a_N$, задавайки най-много $Q$ въпроса.

\section*{Детайли по имплментацията}
Вашата програма трябва да съдържа в началато си \code{\#include "topsecret.h"}.
\\
Вашата програма трябва да реализира следната функция:
\noskip
\begin{itemize}
    \item \code{int find\_position(int N, int Q)}
    \begin{itemize}
        \item \code{N}: номерът на служителя, за когото Ави се представя.
            Общият брой на служителите във фирмата е $N+1$.
        \item \code{Q}: максималният разрешен брой въпроси за съответния тест
        \item Функцията трябва да връща стойността на $a_N$ - позицията, на
            която трябва да застане Ави.
        \item Функцията се вика веднъж за всеки тест.
    \end{itemize}
\end{itemize}
Освен горната функция Вашата програма може да ползва и реализира други вътрешни 
функции и глобални променливи.
Тя трябва \textit{да не съдържа} функция \code{main}.

Вашата програма може да задава въпроси, използвайки следната функция:
\noskip
\begin{itemize}
    \item \code{bool is\_less(int i, int k)}
    \begin{itemize}
        \item \code{i}: индексът на служителя, за чиято позиция в опашката се
            пита.
            Трябва да изпълнява $0 \leq i \leq N-1$.
        \item \code{k}: стойността, с която се сравнява позицията на $i$-тия
            служител.
            Трябва да изпълнява $0 \leq k \leq N$.
        \item Функцията връща \code{true}, ако  и \code{false} в противен случай.
        \item Функцията може да се вика най-много $Q$ пъти за един тест.
    \end{itemize}
\end{itemize}

В случай че нарушите ограниченията за параметрите на функцията, ще получите
съобщение \textit{„Invalid question asked.“}.

Ако извикате \code{is\_less} повече от $Q$ пъти, ще получите съобщение 
\textit{„Too many questions asked.“}.

\section*{Пример}
Грейдърът извършва следното викане на функция:
\noskip
\begin{itemize}
    \item \code{find\_position(3, 15)}: Служителите са общо $4$, Ави е служител
        номер $3$ и програмата може да зададе най-много $15$ въпроса.
\end{itemize}

Програма задава следните въпроси:
\noskip
\begin{itemize}
    \item \code{is\_less(0, 1)} връща \code{true}
    \item \code{is\_less(1, 3)} връща \code{true}
    \item \code{is\_less(1, 2)} връща \code{false}
    \item \code{is\_less(2, 2)} връща \code{false}
\end{itemize}
Единствената редица, която отговаря на горните отговори е $[0, 2, 3, 1]$.
Следователно \code{find\_position} трябва да върне 1.

\section*{Подзадачи}
Във всички тестове $a_i \neq a_j$ за всички $0 \leq i < j \leq N$ и 
$0 \leq a_i \leq N$ за всяко $0 \leq a_i \leq N$.

Тестовете са групирани в подзадачи, като точките за дадена подзадача се
получават само ако програмата Ви премине всички тестове успешно.
\noskip
\begin{itemize}
    \item (10 точки) $1 \leq N \leq 1\ 000$, $Q = 2\ 000\ 000$
    \item (15 точки) $1 \leq N \leq 100\ 000$, $Q = 2\ 000\ 000$,
    \item (15 точки) $1 \leq N \leq 100\ 000$, $Q = 200\ 000$, $a_i < a_{i+1}$
        за всяко $0 \leq i \leq N-2$
    \item (60 точки) $1 \leq N \leq 100\ 000$, $Q = 200\ 000$
\end{itemize}

\section*{Локално тестване}
За да можете да тествате решението си на компютъра си, Ви се предоставят
файловете \code{Lgrader.cpp} и \code{topsecret.h}, които да компилирате
заедно с Вашето решение \code{topsecret.cpp}.

\noskip
\subsection*{Вход}
\noskip
\begin{itemize}
    \item Ред 1: две цели числа $N$ и $Q$
    \item Ред 2: $N + 1$ числа $a_0, a_1, ..., a_{N}$
\end{itemize}
\noskip
\subsection*{Изход}
\noskip
\begin{itemize}
    \item Ред 1:
    \begin{itemize}
        \item \textit{„Output is correct“}, ако програмата е преминала успешно
            теста
        \item \textit{„Invalid question“}, ако програмата е задала въпрос,
            неотговарящ на гореописаните ограничения
        \item \textit{„Too many questions asked“}, ако програмата е извършила
            повече от $Q$ извиквания на \code{is\_less}
        \item \textit{„Output isn't correct“}, ако програмата не е надвишила
            максималния разрешен брой въпроси, но не е намерила правилно
            търсената стойност.
    \end{itemize}
\end{itemize}

\section*{Изпращане на тестове към системата}
Можете да изпращате собствени тестове към системата. Форматът на входните и
изходните данни е същият като този на предоставения локалнен грейдър.
\end{document}
\makeatother
