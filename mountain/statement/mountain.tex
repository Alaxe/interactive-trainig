\documentclass[a4paper,11pt]{article}

\usepackage[a4paper, left=2.5cm, right=2.5cm, bottom=2.5cm, top=3.5cm,
headheight=1cm]{geometry}
\usepackage{polyglossia}
\usepackage{amsmath}
\usepackage{fancyhdr}
\usepackage{titlesec}
\usepackage{enumitem}

\setmainfont[Ligatures=TeX]{Noto Serif}
\newfontfamily\cyrillicfont[Script=Cyrillic]{Noto Serif}

\newcommand{\code}[1]{\texttt{#1}}
\newcommand{\noskip}{\vspace{-\parskip}}

\setmainlanguage{bulgarian}
\PolyglossiaSetup{bulgarian}{indentfirst=true}

\setlength{\parindent}{0em}
\setlength{\parskip}{1em}
\titlespacing{\section}{0pt}{0pt}{0pt}
\setlist{nosep}

\renewcommand{\labelitemi}{$\bullet$}
\renewcommand{\labelitemii}{$\circ$}
\renewcommand{\thesection}{}

\pagestyle{fancy}
\fancyhf{}
\rfoot{\thepage}
\chead{\uppercase{Национална школа по информатика \\Стара Загора, 24-30 август
2017 г.}}
\renewcommand{\headrulewidth}{0pt}

\title{Планина}
\author{Александър Кръстев}
\date{}

\makeatletter
\begin{document}
{%
    \centering \LARGE 
    \textbf{\@title}
    \par
}
%\vspace{0.5cm}

Ави е на поход по планинска пътека, раделена на $N$ участъка.
Надморските височини на участъците се характеризират от редицата 
$a_0, a_1, ..., a_{N-1}$.
Макар и да е в добра физическа форма, Ави е поела по полегата пътека - 
такава, че разликите във височините на всеки два съседни участъка не надвишава
$1$, т.е. $|a_{i+1} - a_i| \leq 1$ за всяко $0 \leq i \leq N-2$.

Момичето иска да изпрати на приятеля си Боби редицата $a$.
Ако сте ходили на планина обаче, ще знаете, че обхватът там (и съответно
мобилният интернет на Ави) е ужасен. 
Поради тази причина тя не може да изпрати повече от $M_{max}$ бита на приятеля
си.

Вашата задача е да напишете \textit{две програми}, чрез които Ави да може да
изпрати редицата $a$ на Боби, използвайки най-много $M_{max}$ бита.
\begin{itemize}
    \item Първата програма получава редицата $a$ и я кодира в $M$ бита
        $b_0, b_1, ..., b_{M-1}$.
    \item Втората програма получава генерираните от първата програма битове $b$
        и дължината на пътеката $N$ и трябва да възстанови оригиналната редица
        $a$.
\end{itemize}

\section*{Детайли по имплментацията}
Трябва да изпратите два сорс файла \code{avi.cpp} и \code{bobi.cpp}, реализиращи
двете програми.

Освен долните функции, Вашите програми могат да ползват и реализират други
вътрешни функции и глобални променливи.
Тъй като изпратените от Вас два сорс файла ще бъдат компилирани заедно с грейдър
в една програма, тези вътрешни функции и глобални променливи трябва да бъдат
декларирани с ключовата дума \code{static}. 

За всеки тест така компилираната програма ще бъде изпълнена два пъти - веднъж
изпълнявайки програмата на Ави и веднъж тази на Боби.
\noskip
\subsection*{avi.cpp}
\noskip
В началото си Вашата програма \code{avi.cpp} трябва да съдържа 
\code{\#include "avi.h"}.\\
Вашата програма \code{avi.cpp} трябва да реализира следната функция:
\noskip
\begin{itemize}
    \item \code{std::vector<bool> encode(std::vector<int> a, int m\_max)}
    \begin{itemize}
        \item \code{a}: редицата от височините на участъците на пътеката
        \item \code{m\_max}: максималният брой битове $M_{max}$, които Ави може
            да изпрати на Боби.
        \item Функцията трябва да връща редица $b$ с дължина $M \leq M_{max}$ -
            съобщението, което Ави изпраща на Боби.
        \item Функцията се вика веднъж за всеки тест.
    \end{itemize}
\end{itemize}

В случай, че вашата функция \code{encode} върне редица с повече от $M_{max}$
бита, ще получите съобщение „Output isn’t correct.“.

\noskip
\subsection*{bobi.cpp}
\noskip
В началото си Вашата програма \code{bobi.cpp} трябва да съдържа 
\code{\#include "bobi.h"}.\\
Вашата програма \code{bobi.cpp} трябва да реализира следната функция:
\noskip
\begin{itemize}
    \item \code{std::vector<int> decode(std::vector<bool> b, int N)}
    \begin{itemize}
        \item \code{b}: съобщението, чрез което вашата програма \code{avi.cpp}
            е кодирала $a$
        \item \code{N}: броя участъци в пътеката.
        \item Функцията трябва да връща редица $а$ с дължина $N$ - височините
            на участъците на пътеката, които е получила вашата програма
            \code{avi.cpp}.
        \item Функцията се вика веднъж за всеки тест.
    \end{itemize}
\end{itemize}
В случай, че вашата функция \code{decode} не върне редица $a$, различна от
зададената, ще получите съобщение „Output isn’t correct.“.

\section*{Пример}
Грейдърът извършва следното извикване на функция от \code{avi.cpp}:
\noskip
\begin{itemize}
    \item \code{encode([0, 1, 1, 0, 1], 20)}
        В случая $a_i \leq 1$ за всяко $0 \leq i \leq N-1$ и всяка височина може
        да се запише с един бит.
        Функцията връща редицата $1, 0, 0, 1, 0$.
\end{itemize}
След това грейдърът се пуска наново и извършва следното извикване на функция от
\code{bobi.cpp}:
\noskip
\begin{itemize}
    \item \code{decode([1, 0, 0, 1, 0], 5)}
        Функцията трябва да върне $0, 1, 1, 0, 1$ - оригиналната редица от
        височини $a$.
\end{itemize}

\section*{Подзадачи}
Във всички подзадачи $0 \leq a_i \leq 10^9$ за всяко $0 \leq i \leq N - 1$.

\begin{enumerate}
    \item (25 точки) $1 \leq N \leq 1000$, $M_{max}=200\ 000$
    \item (25 точки) $1 \leq N \leq 100\ 000$, $M_{max}=210\ 000$
    \item (25 точки) $1 \leq N \leq 100\ 000$, $M_{max}=170\ 000$,
        $|a_{i + 1} - a_i| = 1$ за всяко $0 \leq i \leq N-2$.
    \item (25 точки) $1 \leq N \leq 100\ 000$, $M_{max}=170\ 000$
\end{enumerate}

\section*{Локално тестване}
За да можете да тествате решението си на компютъра си, Ви се предоставят
файловете \code{Lgrader.cpp}, \code{avi.h} и \code{bobi.h}, които да компилирате
заедно с Вашите сорсове \code{avi.cpp} и \code{bobi.cpp}. 

За разлика от грейдъра на системата, локалният грейдър изпълнява функциите 
\code{encode} и \code{decode} наведнъж.
\noskip
\subsection*{Вход}
\noskip
\begin{itemize}
    \item Ред 1: две цели числа $N$ и $M_{max}$
    \item Ред 2: $N$ числа $a_0, a_1, ..., a_{N-1}$
\end{itemize}
\noskip
\subsection*{Изход}
\noskip
\begin{itemize}
    \item Ред 1: редицата $b$, резултатът от Вашата функция \code{encode}
    \item Ред 2: Нека означим дължината на $b$ с $M$.
    \begin{itemize}
        \item Ако $M \leq M_{max}$, редица от $N$ числа,
            резултатът от Вашата функция \code{decode}
        \item Ако $M > M_{max}$, „Too many bits used“.
    \end{itemize}
\end{itemize}

\section*{Изпращане на тестове към системата}
За съжаление изпращането на тестове към системата не работи в момента.
%Можете да изпращате собствени тестове към системата. Форматът на входните данни
%е същият като този на предоставения локалнен грейдър. Изходните данни се състоят
%само от \textit{Ред 2} от изхода на локалния грейдър.
\end{document}
\makeatother
